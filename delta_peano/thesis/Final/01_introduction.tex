\section{Introduction}

\marginpar{\footnotesize What is DEM. Why does DEM matter (application areas)?}
Lorem ipsum. Lorem ipsum. Lorem ipsum. Lorem ipsum. Lorem ipsum. Lorem ipsum.
Lorem ipsum. Lorem ipsum. Lorem ipsum. Lorem ipsum. Lorem ipsum. Lorem ipsum. 

\marginpar{\footnotesize There are two important shortcomings of many codes:
only spheres and too few of them. Furthermore, many codes suffer if spheres
are of different orders of magnitudes. Why does it make sense to tackle this?}
Lorem ipsum. Lorem ipsum. Lorem ipsum. Lorem ipsum.
Lorem ipsum. Lorem ipsum. Lorem ipsum. Lorem ipsum. Lorem ipsum. Lorem ipsum. 

\marginpar{\footnotesize Major objective/contribution of this paper.}
Upscaling a DEM code with respect to particle count and machine size 
challenges the objective to work with triangulated particles from a vast range
of particle sizes.
DEM codes spend a majority of their compute time in collision detection.
\marginpar{\footnotesize Tomek: citation.} 
This phase becomes significantly more complicated if we switch from
sphere-to-sphere checks to the comparison of billions of triangles.
Geometric comparisons suffer from poor SIMDability if realised
straightforwardly.
Multiple contact points between any pair of particles may exist, and it is
impossible to predict statically how this computational workload distributes
between ranks if particles are distributed among ranks. 
The distribution itself is non-trivial if there are particles from a vast range
of scales, which in turn again makes the contact point relation more complicated
than for particle sets of (roughly) the same size.
Finally, the amount of data to be exchanged per particle is higher
than for spherical data where a centre and a radius describe the object of interest. 
The computational work thus increases significantly with any increase of
particle counts.
The scalability of this computational work in turn is not given by construction.
The present paper introduces realisation idioms of a triangle-based DEM code
that fuses efficiency and scalability on all levels of the machine architecture spanning from
vectorisation and memory access characteristics over manycore to inter-node
data exchange.
It shows that the upcoming massively parallel era will allow us to
simulate non-sperical DEM formulations with unpreceded speed.   

\marginpar{\footnotesize What do we do and how does it differ/fit to others'
work?}
Lorem ipsum. Lorem ipsum. Lorem ipsum. Lorem ipsum. Lorem ipsum. Lorem ipsum.
Lorem ipsum. Lorem ipsum. Lorem ipsum. Lorem ipsum. Lorem ipsum. Lorem ipsum. 

\marginpar{\footnotesize Shortcomings and limitations of the present approach.}
Lorem ipsum. Lorem ipsum. Lorem ipsum. Lorem ipsum. Lorem ipsum. Lorem ipsum.
Lorem ipsum. Lorem ipsum. Lorem ipsum. Lorem ipsum. Lorem ipsum. Lorem ipsum. 


- Non-spherical particles reduce efficiency significantly [78,104] in Samiei
- Analytical shape functions besides spheres [41,56,61,118] in Semiei und er
selbst
- Generic shape composition with spheres [75] und Samiei selber

The remainder of the paper is organised as follows: 
We start from a brief sketch the overall DEM simulation with
explicit time stepping in Section \ref{section:algorithm}. 
A multiscale grid meta data structure (Section \ref{section:grid}) helps us to
reduce the algorithms complexity to linear.
Starting from this, the main part of the contribution studies a proper data
layout of particle and collision data (Section \ref{section:vectorisation}) that is well-suited for
vectorisation, allows us to exploit multi- and manycore
architectures in various ways (Section \ref{section:shared-memory}) as well as
classic parallel architectures via MPI and domain decomposition (Section
\ref{section:domain-decomposition}).
In Section \ref{section:benchmarks}, we use all algorithmic ingredients to run some benchmarks
that reveal the potential and impact of a technology supporting non-spherical
particles efficiently, before we present performance studies (Section
\ref{section:results}).
A brief summary and an outlook close the discussion.
